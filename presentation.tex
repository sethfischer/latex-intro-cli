% rubber directives
% rubber: depend helloworld/makefile
% rubber: depend fragments/pdflatex.fragment.php

\documentclass[compress]{beamer}

\usepackage{alltt}
\usepackage{booktabs}
\usepackage{graphicx}
\usepackage{listings}
\usepackage[absolute,overlay]{textpos}
\usepackage[utf8]{inputenc}
\usepackage{multicol}
\usepackage{tikz}


\usetikzlibrary{shapes,arrows}


% beamer configuration
\title{An Introduction to \LaTeX{}}
\subtitle{Creating PDF documents with command-line tools}
\author[S. Fischer]{Seth Fischer}
\institute[github.com/sethfischer]{
    \href{https://github.com/sethfischer/}{github.com/sethfischer}
}
\usetheme{Luebeck}
\setbeamertemplate{caption}[numbered]


% listings package global configuration
\lstset{
    language={[LaTeX]TeX}
}


% define commands

% typeset system paths as mono-spaced font
\newcommand{\syspath}[1] {{\tt #1}}

% typeset package names as mono-spaced font
\newcommand{\pkgname}[1] {{\tt #1}}

% place note in lower left-hand corner of frame
\setbeamercolor{framesource}{fg=gray}
\setbeamerfont{framesource}{size=\tiny}
\newcommand{\source}[1]{\begin{textblock*}{7cm}(0.25cm,8.6cm)
    \begin{beamercolorbox}[ht=0.5cm,left]{framesource}
        \usebeamerfont{framesource}\usebeamercolor[fg]{framesource} {#1}
    \end{beamercolorbox}
\end{textblock*}}

\def\bottom#1#2{\hbox{\vbox to #1{\vfill\hbox{#2}}}}


\begin{document}

\begin{frame}
    \titlepage
\end{frame}

\begin{frame}{Outline}
    %\begin{multicols}{2}
        \tableofcontents[hideallsubsections]
    %\end{multicols}
\end{frame}

\begin{frame}{Example markup}
    \begin{itemize}
        \item Many of the following markup examples are taken from the \LaTeX{} 
            source for this presentation.
        \item The complete hello world example can be found in the 
            \syspath{helloworld} directory.
        \item \LaTeX{} markup fragments may be found in the \syspath{fragments} 
            directory.
        \item Source code for this presentation is available at 
            \href{https://github.com/sethfischer/latex-intro-cli}
            {github.com/sethfischer/latex-intro-cli}
    \end{itemize}
\end{frame}


\section{Introduction}

\subsection{\LaTeX{}}

\begin{frame}{What is \LaTeX{}?}
    \LaTeX{} is a high-quality typesetting system and markup language used to 
    describe document structure.

    \begin{itemize}
        \item Pronounced ``Lah-tec'' or ``Lay-tec''.
        \item Developed in 1985 by Leslie Lampor.
        \item \LaTeX{} is a set of macros for Donald E. Knuth's \TeX{} 
            typesetting language.
    \end{itemize}
\end{frame}

\begin{frame}{What is \LaTeX{} used for?}
    \LaTeX{} contains features for:

    \begin{itemize}
        \item Typesetting journal articles, technical reports, books, and slide 
            presentations.
        \item Control over large documents containing sectioning, 
            cross-references, tables and figures.
        \item Typesetting of mathematical formula.
        \item Automatic generation of bibliographies and indices.
    \end{itemize}

\end{frame}


\subsection{Installation}

\begin{frame}[fragile]{Installation}

    \begin{block}{Debian-based distributions}
\begin{lstlisting}
# apt-get install texlive-full
\end{lstlisting}
        {
            \footnotesize
            \pkgname{texlive-full} is a meta-package which pulls in all the 
            relevant texlive packages.
        }
    \end{block}

    \begin{block}{Manual installation}
        See \href{http://www.tug.org/texlive/}{http://www.tug.org/texlive/}
    \end{block}

    \begin{block}{In-browser}
        Visit \href{https://www.writelatex.com/}{https://www.writelatex.com/}
    \end{block}

\end{frame}

\subsection{Hello world}

\begin{frame}[fragile]{Hello world}
    \begin{columns}[T]
        \column{0.5\textwidth}
            \begin{block}{Markup}
                \lstinputlisting{helloworld/helloworld.tex}
            \end{block}
        \column{0.5\textwidth}
            \begin{block}{Output}
                \setlength\fboxsep{0pt}
                \setlength\fboxrule{0.5pt}
                \begin{center}
                    \colorbox{white}{\fbox{
                        \includegraphics[height=4.5cm]{helloworld/helloworld.pdf}
                    }}
                \end{center}
            \end{block}
    \end{columns}
\end{frame}

\begin{frame}[fragile]{Compilation with pdflatex}

    \begin{block}{Terminal}
\begin{lstlisting}
$ pdflatex helloworld
\end{lstlisting}
    \end{block}

    \begin{block}{Makefile}
        \lstinputlisting[
            language={[gnu]make},
            showtabs=true
        ]{helloworld/makefile}
    \end{block}

\end{frame}

\subsection{Basics}

\begin{frame}{Reserved characters}
    \begin{center}
        \begin{tabular}{llr}
            \toprule
            Character           & Escaped                           \\
            \midrule
            \#                  & \textbackslash{}\#                \\
            \midrule
            \$                  & \textbackslash{}\$                \\
            \midrule
            \%                  & \textbackslash{}\%                \\
            \midrule
            \^{}                & \textbackslash{}\^{}\{\}          \\
            \midrule
            \&                  & \textbackslash{}\&                \\
            \midrule
            \{                  & \textbackslash{}\{                \\
            \midrule
            \}                  & \textbackslash{}\{                \\
            \midrule
            \~{}                & \textbackslash{}\~{}              \\
            \midrule
            \textbackslash{}    & \textbackslash{}textbackslash\{\} \\
            \bottomrule
        \end{tabular}
    \end{center}
\end{frame}

\begin{frame}[fragile]{Groups and environments}
    \framesubtitle{The building blocks}
    \begin{block}{Groups}
\begin{lstlisting}
{
    \bf This is bold.
}
This is no longer bold.
\end{lstlisting}
    Used to restrict the range of action of a command.
    \end{block}
    \begin{block}{Environments}
\begin{lstlisting}
\begin{environmentname}
    text to be influenced
\end{environmentname}
\end{lstlisting}
    Affect a wider part of the document.
    \end{block}
\end{frame}


\begin{frame}[fragile]{Commands}
    \begin{block}{Syntax}
\begin{lstlisting}
\commandname[opt1,opt2,...]{arg1}{arg2}...
\end{lstlisting}
    \end{block}
    \begin{block}{Example}
\begin{lstlisting}
\includegraphics[height=4.5cm]{helloworld.pdf}
\end{lstlisting}
    \end{block}

\end{frame}

\begin{frame}[fragile]{Comments (and line breaks)}
    \begin{columns}[T]
        \column{0.5\textwidth}
            \begin{block}{Markup}
                \lstinputlisting{fragments/comment.fragment.tex}
            \end{block}
        \column{0.5\textwidth}
            \begin{block}{Output}
                % LaTeX comments
% and line breaks
Line zero \\
Line one \\ % comment 1
% comment 2
Line two%
    , three
    and four.

            \end{block}
    \end{columns}
\end{frame}

\section{Markup}

\subsection{Semantic structure}

\begin{frame}[fragile]{Global structure}
    Every input file must contain the following commands:
\begin{lstlisting}
\documentclass[options]{class}
\usepackage{...}

\begin{document}
...
\end{document}
\end{lstlisting}
\end{frame}

\subsection{Classes}

\begin{frame}{Document classes}
    Some of the more common document classes are:
    \begin{description}
        \item[article] For articles in scientific journals, presentations, 
            short reports, program documentation, \ldots
        \item[minimal] As small as it can get. Only sets a page size and a base 
            font.
        \item[report] For longer reports, thesis, \ldots
        \item[book] For books.
        \item[memoir] Based on the book class, but can be used to create any 
            kind of document.
        \item[letter] For writing letters.
        \item[beamer] For writing presentations (such as this).
    \end{description}
\end{frame}

\subsection{Packages}

\begin{frame}{Packages}
    Packages extend the functionality of the selected document class. Among 
    others, this presentation uses:
    \begin{description}
        \item[booktabs] Style tables.
        \item[graphicx] Include images.
        \item[hyperref] Create hyperlinks.
        \item[listings] Syntax highlighting of code listings.
        \item[textpos] Locate text at absolute positions on a page.
    \end{description}
\end{frame}

\subsection{Common elements}

\begin{frame}[fragile]{Chapters and sections}
    \begin{center}
        \begin{tabular}{llr}
            \toprule
            Level   & Command                   \\
            \midrule
            -1      & \verb|\part{}|            \\
            \midrule
            0       & \verb|\chapter{}|         \\
            \midrule
            1       & \verb|\section{}|         \\
            \midrule
            2       & \verb|\subsection{}|      \\
            \midrule
            3       & \verb|\subsubsection{}|   \\
            \midrule
            4       & \verb|\paragraph{}|       \\
            \midrule
            5       & \verb|\subparagraph{}|    \\
            \bottomrule
        \end{tabular}
    \end{center}
\end{frame}

\begin{frame}[fragile]{Paragraphs}
    \begin{columns}[T]
        \column{0.5\textwidth}
            \begin{block}{Markup}
                \lstinputlisting{fragments/paragraphs.fragment.tex}
            \end{block}
        \column{0.5\textwidth}
            \begin{block}{Output}
                Leave a blank line 
between paragraphs.

Second paragraph.



Third paragraph.

            \end{block}
    \end{columns}
\end{frame}

\begin{frame}[fragile]{Lists}
    \begin{columns}[T]
        \column{0.5\textwidth}
            \begin{block}{Markup}
                \lstinputlisting{fragments/lists.fragment.tex}
            \end{block}
        \column{0.5\textwidth}
            \begin{block}{Output}
                \begin{enumerate}
 \item The first item
 \begin{itemize}
  \item First sub item
  \item Second sub item
 \end{itemize}
 \item The second item
 \item The third item
\end{enumerate}

            \end{block}
    \end{columns}
\end{frame}

\begin{frame}{Tables}
    \begin{columns}[T]
        \column{0.5\textwidth}
            \begin{block}{Markup}
                \lstinputlisting{fragments/tabular.fragment.tex}
            \end{block}
        \column{0.5\textwidth}
            \begin{block}{Output}
                \input{fragments/tabular.fragment.tex}
            \end{block}
    \end{columns}
\end{frame}

\begin{frame}{Images}
    \begin{columns}[T]
        \column{0.7\textwidth}
            \begin{block}{Markup}
                \lstinputlisting{fragments/includegraphics.fragment.tex}
            \end{block}
        \column{0.3\textwidth}
            \begin{block}{Output}
                \begin{figure}
\centering
    \includegraphics
        [height=3cm]
        {images/tux.png}
    \caption{
        Tux, as originally 
        drawn by Larry Ewing
    }
    \label{fig:tux}
\end{figure}

            \end{block}
    \end{columns}
\end{frame}

\section{Macros}

\begin{frame}[fragile]{New commands}
    Define custom commands with \textbackslash{}newcommand.
\begin{lstlisting}
\newcommand{name}[num]{definition}
\end{lstlisting}

    \begin{block}{Markup}
\begin{lstlisting}
% typeset system paths as mono-spaced font
\newcommand{\syspath}[1] {{\tt #1}}
...
The config directory is \syspath{/etc/apache2}
\end{lstlisting}
    \end{block}

    \begin{block}{Output}
        The config directory is \syspath{/etc/apache2}
    \end{block}
\end{frame}

\section{Compilation}

\subsection{The compilation process}

\begin{frame}{The compilation process}

    \tikzstyle{group} = [
        shape=rectangle,
        fill=lightgray,
        draw=black
    ]

    \tikzstyle{source} = [
        shape=rectangle,
        rounded corners,
        minimum width=1.5cm,
        minimum height=0.5cm,
        draw=black,
        fill=white,
        text centered
    ]

    \tikzstyle{output} = [
        shape=rectangle,
        rounded corners,
        minimum height=0.6cm,
        minimum width=2cm,
        draw=black,
        fill=white,
        text centered
    ]

    \begin{center}
        \begin{tikzpicture}[scale=1,transform shape]
            \tikzstyle{every node}=[font=\small]
            \tikzset{>=triangle 45}

            \node (outputformats)
                [group,minimum width=10cm,minimum height=2.5cm,]
                at (0,0)
                {};

            \node (sourceformats)
                [group,minimum width=2.5cm,minimum height=3cm,]
                at (0,4)
                {};

            \node (tex) at (0,1) {Output formats};
            \node (tex) at (0,2.75) {Source formats};

            \node (tex) [source] at (0,5) {\TeX};
            \node (latex) [source] at (0,3.5) {\LaTeX{}};
            \node (postscript) [output] at (0,0) {PostScript};
            \node (dvi) [output] at (3.75,0) {DVI};
            \node (pdf) [output] at (-3.75,0) {PDF};

            \draw [<->] (-2.75,0) -- (-1,0)
                node [above,midway] {\tt ps2pdf}
                node [below,midway] {pdf2ps};

            \draw [<-] (1,0) -- (2.75,0)
                node [above,midway] {\tt dvi2ps};

            \draw [<-] (-3.75,-0.3) -- (-3.75,-1) -- (3.75,-1)
                node [above,midway] {\tt dvipdfm} -- (3.75,-0.3);

            \draw [<-] (-4.25,0.3) -- (-4.25,5) -- (-0.75,5)
                node [above,midway] {\tt pdftex};

            \draw [<-] (-3.25,0.3) -- (-3.25,3.5) -- (-0.75,3.5)
                node [above,midway] {\tt pdflatex};

            \draw [<-] (4.25,0.3) -- (4.25,5) -- (0.75,5)
                node [above,midway] {\tt tex};

            \draw [<-] (3.25,0.3) -- (3.25,3.5) -- (0.75,3.5)
                node [above,midway] {\tt latex};

        \end{tikzpicture}
    \end{center}
    \source{\copyright Alessio Damato. Redrawn as TikZ image by S. Fischer}
\end{frame}

\subsection{Ancillary files}


\begin{frame}{Ancillary files}
    \begin{itemize}
        \item \TeX{} compilers use a single pass.
        \item Ancillary files store temporary data to be used in the next 
            compilation.
        \item Information regarding the previous compiler pass is located in a 
            \syspath{*.log} file.
    \end{itemize}
\end{frame}

\begin{frame}{Ancillary files}
    This presentation generates the following ancillary files:
    \begin{description}
        \item[\syspath{*.aux}] Transports information from one compiler run to 
            the next.
        \item[\syspath{*.nav}] Beamer class specific.
        \item[\syspath{*.out}] Hyperref package specific (PDF bookmarks \ldots).
        \item[\syspath{*.snm}] Beamer class specific.
        \item[\syspath{*.toc}] Used to produce table of contents.
        \item[\syspath{*.vrb}] Beamer class specific.
    \end{description}
\end{frame}

\begin{frame}[fragile]{Cross-references}
    \begin{itemize}
        \item Compiler must be executed a number of times to resolve 
            cross-references.
        \item Documents with any of the following (among others) will require 
            multiple runs of pdflatex:
        \begin{itemize}
            \item table of contents
            \item table of figures
            \item references to figures
            \item page~X~of~Y
        \end{itemize}
    \end{itemize}
    \begin{block}{Rerun pdflatex to resolve cross-references}
\begin{alltt}\scriptsize
$ pdflatex inputfile | grep "LaTeX Warning"
{\bf{LaTeX Warning: Label(s) may have changed. Rerun to get cross-references right.}}
$ pdflatex inputfile | grep "LaTeX Warning"
$ 
\end{alltt}
    \end{block}

\end{frame}


\subsection{Compilation tools}

\begin{frame}{latex}
    \begin{center}
        {\Huge \LaTeX{} $\xrightarrow{\tt latex}$ DVI}
    \end{center}
    \begin{block}{}
        latex executable calls the tex executable with \LaTeX{} initialisation 
        files, reads a \LaTeX{} \syspath{.tex} file and creates a 
        \syspath{.dvi}.
    \end{block}
    \begin{center}
        DVI $\xrightarrow{\tt dvi2pdf}$ PDF \\
    \end{center}

    Supported image formats:
    \begin{itemize}
        \item EPS
    \end{itemize}
\end{frame}

\begin{frame}{pdflatex}
    \begin{center}
        {\Huge \LaTeX{} $\xrightarrow{\tt pdflatex}$ PDF}
    \end{center}
    \begin{block}{}
        pdflatex executable calls the pdftex executable with \LaTeX{} 
        initialisation files, reads a \LaTeX{} \syspath{.tex} file and creates 
        a \syspath{.pdf}.
    \end{block}
    Supported image formats:
    \begin{itemize}
        \item JPG
        \item PNG
        \item PDF
    \end{itemize}
\end{frame}

\begin{frame}{make}
    \begin{itemize}
        \item make can be used to automate the compilation of a \LaTeX{} 
            project.
        \item It can be difficult to automate multiple compiler passes without 
            external scripts.
        \item There are a number of generic makefiles available for compiling 
            \LaTeX{} projects.
    \end{itemize}
\end{frame}

\begin{frame}[fragile]{rubber}
    \framesubtitle{
        \href{https://launchpad.net/rubber}{https://launchpad.net/rubber}
    }

    \begin{itemize}
        \item A wrapper for latex and companion programs (written in Python).
        \item Compiles the project enough times so that all references are 
            defined.
        \item Runs BibTeX to manage bibliographic references.
        \item The command rubber-pipe reads \LaTeX{} source from stdin and 
            dumps the compiled document on stdout.
    \end{itemize}
\end{frame}

\begin{frame}[fragile]{rubber}
    \framesubtitle{Directives}

    \begin{itemize}
        \item rubber extracts information from the \LaTeX{} source.
        \item Sometimes rubber does not know what to do.
        \item rubber can be instructed with directives which are standard 
            comments in the \LaTeX{} source.
    \end{itemize}

    \lstinputlisting[
        lastline=3,
    ]{presentation.tex}
\end{frame}

\begin{frame}{latexmk}
    \framesubtitle{
        \href{http://www.ctan.org/pkg/latexmk/}{http://www.ctan.org/pkg/latexmk/}
    }

    \begin{itemize}
        \item Perl script for running latex the correct number of times 
            to resolve cross-references, etc.
        \item Similar in function to rubber.
    \end{itemize}
\end{frame}

\begin{frame}{Output format (other than PDF)}
    Many tools exist for converting \LaTeX{}/DVI/PDF files to other formats:
    \begin{description}
        \item[tex4ht] HTML, XML, braille, etc., optionally using MathML.
        \item[latex2rtf] Rich Text Format (first step in converting to an Open 
            Document Format).
        \item[pdf2svg] Scalable Vector Graphics.
        \item[detex] Plain text.
        \item[catdvi] Plain text, retaining formatting. Pipe through 
            {\tt fmt -u} to remove justification.
    \end{description}
\end{frame}

\section{Tips}

\begin{frame}[fragile]{Tips}
    \begin{block}{Spell checking}
        aspell in tex mode

        See the \syspath{spellcheck} directory for an example.
    \end{block}
    \begin{block}{Collaboration}
        Use a version control system such as git.
    \end{block}
    \begin{block}{Compressing documents produced with pdflatex}
\begin{lstlisting}[language=bash]
$ gs -dBATCH -dNOPAUSE -q -sDEVICE=pdfwrite \
-sOutputFile=out.pdf in.pdf
\end{lstlisting}
    \end{block}
\end{frame}

\begin{frame}{Dynamic PDF documents in a web application}
    \framesubtitle{Using rubber-pipe}
    \lstinputlisting[
        language=php,
        frame=none,
        showstringspaces=false,
        basicstyle=\tiny\ttfamily
    ]{fragments/pdflatex.fragment.php}
\end{frame}

\appendix

\section{Resources}

\begin{frame}{Further reading}
    \begin{thebibliography}{Wikibooks LaTeX, 2013}

        \bibitem[Oetiker, 2011]{Oetiker2011}
            Tobias Oetiker et al.
            \newblock {\em The Not So Short Introduction to \LaTeXe}.
            \newblock \href{http://tobi.oetiker.ch/lshort/lshort.pdf}
                {http://tobi.oetiker.ch/lshort/lshort.pdf}

        \bibitem[Wikibooks LaTex, 2013]{WikibooksLaTex2013}
            Wikibooks
            \newblock {\em \LaTeX 
                \footnote{
                    Some of the markup examples in this presentation were 
                    inspired, or reproduced verbatim, from 
                    http://en.wikibooks.org/wiki/LaTeX.
                }%
            }.
            \newblock \href{http://en.wikibooks.org/wiki/LaTeX}
                {http://en.wikibooks.org/wiki/LaTeX}

    \end{thebibliography}
\end{frame}

\end{document}

